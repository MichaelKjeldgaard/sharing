Denne synopsis er en del af det indledende afklarende arbejde der
udføres før der tages beslutning om udarbejdelse af en
referencearkitektur. Formålet er at konkretisere et muligt indhold med
henblik på udpegning af interessenter samt at afgrænse opgaven i forhold
til øvrige aktiviteter. Synopsis vil, på kortest mulige form, giver et
overblik over strukturen og indholdet af den endelige arkitektur.
Synopsen er ikke et gennemtænkt bud på den endelige løsning, men skal
udtale sig om retning og afprøve rammerne for det videre arbejde.

\section{Introduktion}\label{introduktion}

\subsection{Formål}\label{formuxe5l}

Referencearkitekturen understøtter udviklingen af it-systemer - der
anvender og (sammenstiller) registeroplysninger til sagsbehandling eller
selvbetjening - der sender eller modtager meddelelser fra andre
it-systemer

\subsection{Scope}\label{scope}

\begin{itemize}
\tightlist
\item
  Aktiviteter under Digitaliseringsstrategien
\item
  Offentlige {[}autoritative?{]} registre {[}vedrørende personer og
  virksomheder?{]} {[}, hvortil der er knyttet rettigheder?{]}
\item
  Kun anvendelse, udstilling {[}og fejlrettelser?{]} udenfor
  registerejer
\end{itemize}

\subsection{Centrale begreber}\label{centrale-begreber}

\begin{itemize}
\tightlist
\item
  Register, registerejer, dataansvarlig, dataanvender, den registrerede.
\item
  Grunddata,
\item
  Dokument,
\item
  Afsender, modtager, meddelelse
\end{itemize}

\subsection{Anvendelse}\label{anvendelse}

\begin{itemize}
\tightlist
\item
  bruges sprog til at formulere en fælles strategi
\item
  bruges som reference ved løsningsbeskrivelser
\end{itemize}

\subsection{Tilblivelse og governance}\label{tilblivelse-og-governance}

Denne version er skrevet\ldots{}.og rettet mod\ldots{} Endelig
godkendelse hos SDA

\subsection{Metoderamme}\label{metoderamme}

\begin{itemize}
\tightlist
\item
  OIO referencearkitektur, men også EIRA enablers.
\end{itemize}

\subsection{Relation til andre
referencearkitekturer}\label{relation-til-andre-referencearkitekturer}

\begin{itemize}
\tightlist
\item
  Anvender brugerstyring
\item
  Anvendes af selvbetjening og overblik
\end{itemize}

\section{Strategi?}\label{strategi}

\subsection{Vision, mål og
strategier}\label{vision-muxe5l-og-strategier}

\subsection{Forretningsmæssige
tendenser}\label{forretningsmuxe6ssige-tendenser}

\begin{itemize}
\tightlist
\item
  Ensretning og nationale indsatser
\item
  Data øget værdi for organisationer
\item
  Øget bevågenhed omkring beskyttelse af privatliv
\item
  Øget opmærksomhed om håndtering af personlige oplysninger
\item
  Mængden af oplysninger der håndteres stiger -b Grænseoverskridende
  services
\end{itemize}

\subsection{Teknologiske tendenser}\label{teknologiske-tendenser}

\begin{itemize}
\tightlist
\item
  øget central standardisering af begreber, datamodeller og grænseflader
\item
  Flere og mere forskelligartede enheder forbundet til netværket
\item
  Øgede forventninger til brugervenlighed af offentlige digitale
  services
\item
  Mængden af tilgængelige oplysninger vokser
\item
  Arkitekturvision for anvendelse og udstilling
\item
  Intergrated Service Delivery
\item
  ''Once only''
\item
  ''Ineroperability/Samarbejdende infrastrukturer / Økosystem af fælles
  løsninger?''
\item
  ''Valgfri for anvender mellem flere tekniske udbydere af samme
  oplysninger''
\end{itemize}

\subsection{Målsætning}\label{muxe5lsuxe6tning}

{[}beskriv målsætninger i eksisterende aftaler og strategier{]}
Strategiske principper F1: Autoritative register med henvisninger til
andre registre F2: Ansvar for begrænsning af adgang ligger hos
registerejer F3: Let at komme med forslag til rettelser I1: Fælles
referenceinformationsmodel I2: Dokument-princip (attester mv.)? A1:
Onlineopslag i sagsbehandling og selvbetjening A2: Log adgang A3: Adgang
til og fra internationale registre sker gennem national gateway T1:
Central fuldmagt/rettighedsstyring T2: Multi-flavour-api

\subsection{Værdiskabelse}\label{vuxe6rdiskabelse}

Mindre besvær for borger og virksomheder ved brug af digitale services
Simplere arbejdsgange og mere potentiale for automatisering hos
myndigheder {[}og virksomheder{]} Understøtte transparens og bevare
tillid til registre Effektiv systemudvikling (begrænse udfaldsrum,
opsamle best practice)

\begin{description}
\tightlist
\item[Interoperability]
\emph{mål} om sammenhængende services\ldots{} integrated service
delivery
\item[Once-only]
\emph{mål} om at borger og virksomhed kun skal afgive den samme
information til det offentlige en gang\ldots{} (men give lov til
genbrug?)
\item[Transperancy]
\emph{mål} om borger og virksomheder skal kunne se hvilke data der
findes om dem og hvor disse data anvendes
\item[Re-use]
\emph{mål} om genbrug af it med henblik på lavere omkostninger
\end{description}

\section{Forretning}\label{forretning}

\subsection{Opgaver}\label{opgaver}

\begin{itemize}
\tightlist
\item
  Borger og virksomhedsvendte selvbetjeningsløsninger
\item
  Myndigheders sagsbehandling
\item
  Tværgående analyse
\item
  {[}meget generiske{]}
\end{itemize}

\emph{Funktion/proces trin}

Registrering \textasciitilde{} \emph{funktion} hvor oplysninger bringes
på digital form

Datanvendelse \textasciitilde{} \emph{funktion} hvor oplysninger
anvendes i en opgave

Registreret forsendelse \textasciitilde{} \emph{funktion} hvor
meddelelser sendes uafviseligt

\subsection{Aktører, roller og
ansvar}\label{aktuxf8rer-roller-og-ansvar}

\begin{itemize}
\tightlist
\item
  Borger, virksomhed, offentlig myndigheder {[}udenlandske?{]}
\item
  Registerejer, Registeranvender, ''Den som data handler om''.
\item
  Public Service Provider, Private Service Provider
\item
  Service Consumer
\end{itemize}

Registrant \textasciitilde{} \emph{rolle} som bringer oplysninger på
digital form, registrer

\begin{description}
\tightlist
\item[Dataejer]
\emph{rolle} som ejer registreringer/data
\item[Dataanvender]
\emph{rolle} der anvender oplysninger fra et register
\item[Datasubject]
\emph{rolle} som oplysninger handler
\item[Dataindeksejer]
\emph{rolle} som er ansvarlig for opbevaring af metadata
\item[Datadistributør]
\emph{rolle} som er ansvarlig for adgang til data for dataanvendere
\end{description}

\subsection{Mønstre? Generic use
cases}\label{muxf8nstre-generic-use-cases}

\begin{itemize}
\tightlist
\item
  Registrering, (gen)-brug af data, forsendelse
\end{itemize}

\subsection{Tværgående processer
(proces-trin)}\label{tvuxe6rguxe5ende-processer-proces-trin}

\begin{itemize}
\item
  Sagsbehandling {[}sag og dokument{]}, id, belysning, hændelser
\item
  Simpel selvbetjening {[}CBI?{]} Genkende, godkende {[}Krølle om rette
  forkerte oplysninger{]}
\item
  Tværgående selvbetjening
\item
  Indsigt i oplysninger og deres anvendelse
\item
  Sende meddelelse
\item
  Modtage meddelelse
\item
  Tag et dokument med til en anden service provider (der ikke har adgang
  til registre)
\end{itemize}

\subsection{Tjenester}\label{tjenester}

Nødvendige: Dataservice(Register), eDelivery Service, Katalog,
Kontaktregister,, Log(Overblik). Ønskelige: Signering, Indeks Mangler:
Referencedata (Klassifikation), Identitet/brugerstyring

\subsection{Forretningsobjekter}\label{forretningsobjekter}

Meddelelse, Påmindelse, Dokument, URI
Rettighed/hjemmel/samtykke/retskilde Kontekst, anvendelse og afsendelse

Teknik Services (Applikationsroller?, enablers?, capabilities?) Opslag
Meld forslag til korrektur?

\section{Tekniske implementeringer}\label{tekniske-implementeringer}

\subsection{Adgang til data hos anden
myndighed}\label{adgang-til-data-hos-anden-myndighed}

\begin{itemize}
\item
  Direkte adgang, SOA
\item
  Datadistribution, sammenstilling samt adgangskontrol og logning
\item
  Fælles Data og applikations-platform
\item ~
  \subsection{Forsendelse}\label{forsendelse}
\item
  SOA
\item
  Fælles system
\item
  Specific eDelivery
\item
  Generic eDelivery
\end{itemize}

\subsection{Områder for
standardisering/profileringer}\label{omruxe5der-for-standardiseringprofileringer}

(Per mønster?, matrix) - Service Design Guidelines - Access Protocols -
Distribution Protocols - Synchronisation Protocols

\begin{itemize}
\tightlist
\item
  Metadata for opslag/søgning/anvendelse
\item
  Log format
\item
  Identifikation
\item
  Klassifikation af følsomhed
\item
  Klassifikation af anvendelse (sagsbehandling vs analyse)
\item
  Hændelsesbeskeder
\item
  Protokol for flytning af filer, kryptering
\item
  Hjemmel (samtykke, lov)
\item
  Context
\end{itemize}

\subsection{Identifikation af
standarder}\label{identifikation-af-standarder}
